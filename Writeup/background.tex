
\section{Expression Trees}
\label{sec:background}

For the purposes of our experiment, we represent functions as expression
trees, in which each node represents either a variable, some constant, or
some binary operation (addition, subtraction, multiplication or division). Powers
are represented, when applicable, as multplications of variables.

Trees are randomly constructed with a depth limit of eight, and with strictly
downward-pointing references. The root node is always some binary operation,
and subsequent nodes have a 25$\%$ probability of being a constant, a 25$\%$ probability
of being a variable, and a 50$\%$ probability of being a randomly chosen binary operation. The tree is
recursively filled in any direction either until a terminal node is chosen (constant or variable),
or until the depth limit has been reached, in which case the chosen node will always
be either a constant or variable. Trees are evaluated from the top, so that higher nodes
receive higher precedence, and are able to accept one or three variables, depending
on the scenario.

Two important operations must be performable on expression trees for purposes of the
genetic algorithm: crossover and mutation. During crossover, a random downward reference
is picked once from two different trees (below the root), and the subtrees below each are
swapped with the assistance of a placeholder reference. During mutation, a random node in
the tree is picked to be replaced with another node. The caveat is that mutations must
be restricted to categorical similarity. Binary operations may only be swapped with binary
operations and variables or constants can only be swapped with variables or constants, so
that all tree connections remain relevant and intact.